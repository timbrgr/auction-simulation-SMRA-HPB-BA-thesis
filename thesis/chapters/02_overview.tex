% !TeX root = ../main.tex
% Add the above to each chapter to make compiling the PDF easier in some editors.

\chapter{Overview of the Market of Spectrum Auctions}\label{chapter:overview}
This chapter introduces national spectrum auctions to establish their necessity 

\section{Necessity of Spectrum Auctions}
TODO

\section{Examples of National Spectrum Auctions}
The following section will give two examples of national spectrum auctions. The first section describes the German spectrum auction from 2015 in detail. The information provided in this section builds the basis for the auction framework used to simulate this auction (see \autoref{chapter:simulation}). As a second example, the 2014 LTE auction in the United Kingdom will be described.

\subsection{Germany - "Mobiles Breitband Projekt 2016"} \label{sec:MBP16}
The German frequency auction in 2015 
The auction format was an open Simultaneous Multi Round Action (SMRA) %TODO ABB SRMA
(see \autoref{sec:SMRA} %TODO REF
), with "open" referring to its degree of available bidding data during the auction.

ran over 4 weeks / 181 rounds and totalted over EUR 5 billion

\textbf{Overview and market} 

The auctioned spectrum UMFAESST 270 MHz, which was spread over four different spectrum bands 700 MHz, 900 MHz, 1800 MHz as well as 1500 MHz and totaled a sum of EUR 5.081bn, as can be seen in the table below. %TODO ref table

%TODO table of auctioned spectrum

The three main German mobile network operators %TODO ABB MNO
%TODO ABB TEF, DT, VOD
Telefonica, Deutsche Telekom and Vodafone participated as bidders in the auction.
Just shortly before the auction, the German MNO market changed due to the merger of the two operators Telefonica and E-plus, reducing the number of agents in the market.


\textbf{Auction specific mechanisms}






\subsection{UK LTE Auction in 2014}
TODO