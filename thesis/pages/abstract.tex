\chapter{\abstractname}

%%TODO: Abstract
%Over the last two decades, auctions have been well established as a reliable mechanism for allocating all sorts of items and goods to prospective buyers, especially in the case of selling licenses of frequency spectra to businesses providing wireless communication services \cite{Cramton2002}. 
With specific implementations varying world wide, characteristically spectrum auctions comprise of many licenses being auctioned simultaneously. Allowing bidders to submit bids only on single items such as in the well established \textit{Simultaneous Multi-Round Action (SMRA)}, acquiring certain combinations of complementing objects can arise in strategical problems for bidders. Combinatorial auctions try to mitigate this effect by allowing bids on bundles of items, but computational challenges emerge due to possible allocations being in combinatorial magnitude with growing number of items being offered. To reduce computational effort,\textit{ Hierarchical Package Bidding (HPB)} was introduced which arranges items in a hierarchical tree structure.
The effort of this thesis is to explore indications of the eligibility of HPB in the context of spectrums auctions by running simulations of the SMRA and HPB auction formats based on the German spectrum auction of 2015. The results indicate, that HPB might be helpful in enabling bidders to find more efficient overall allocations when choosing an appropriate bidding hierarchy. This improvement came with the reduction of overall revenue as bidder signalling improved. SMRA was superior when a less competitive HPB hierarchy was used, indicating that choosing a suitable hierarchy is key to the success of the auction. 

\newpage
\chapter{Zusammenfassung}
Ein Charakteristikum von Spektrumsauktionen ist die simultane Versteigerung von einer Vielzahl von Lizenzen. Wenn es Bietern nur möglich ist, Angebote auf einzelne Objekte abzugeben, wie es beispielsweise in der wohl etablieren \textit{Simultaneous Multi-Round Action (SMRA)} der Fall ist, treffen Bieter auf strategische Herausforderungen, wenn sie Kombinationen von Objekten mit Komplementaritäten erwerben wollen. Kombinatorische Auktionen versuchen diesen Effekt zu lindern, indem sie Gebote auf Objekt-Bündelungen zulassen. Jedoch steigt die Anzahl möglicher Allokationen im kombinatorischen Maße mit der Anzahl an versteigerten Objekten, wodurch es zu Herausforderungen im Rechenaufwand und der Rechenzeit kommen kann. Um den benötigten Rechenaufwand zu reduzieren, wurde das \textit{ Hierarchical Package Bidding (HPB)} entwickelt, welches Auktionsobjekte in einer hierarchischen Baumstruktur arrangiert. Das Ziel dieser Arbeit ist es, die Eignung von HPB im Kontext von Spektrumsauktionen zu untersuchen, indem Simulationen der SMRA und HPB Auktionsformate auf Basis der deutschen Spektrumsauktion vom Jahr 2015 durchgeführt wurden. Die Ergebnisse deuten darauf hin, dass HPB es Bietern ermöglicht effizientere Allokationen zu finden, gesetzt dem Fall, es wird eine geeignete Hierarchiestruktur verwenden. Diese Verbesserung kam mit einer Reduktion der Umsätze, da die Kommunikation zwischen den Bietern sich verbesserte. Bei der Verwendung von weniger kompetitiven Hierarchiestrukturen war die SMRA überlegen, was darauf hindeutet, dass die Wahl einer geeigneten Hierarchie eine große Bedeutung für den Erfolg einer Auktion hat.